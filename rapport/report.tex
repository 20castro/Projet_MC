\documentclass{article}
\usepackage{amsmath}
\usepackage{amssymb}
\usepackage{amsthm}
\usepackage{thmtools}
\usepackage{cases}
\usepackage{enumitem}
\usepackage{xcolor}
\usepackage{cancel}
\usepackage{cite}

\usepackage{algorithm2e}
\RestyleAlgo{ruled}
\SetKwComment{Comment}{/* }{ */}

\usepackage{geometry}
\geometry{margin=4cm, vmargin=3cm}

\setlength\parindent{0pt}

\title{Asian options pricing}
\author{David Castro \qquad Maxime Leroy}
\date{15 Januray 2024}

\begin{document}
\maketitle

\section*{Introduction}

An Asian option is any option with payoff of the form:
\[
	\left( S_t \right)_{t \in [0, T]} \mapsto g \left( S_T, A_T \right) \quad \text{with} \quad A_T := \frac{1}{T} \int_0^T S_u du
\]
where $\left( S_t \right)_t$ denotes the trajectory of the underlying and $T$ is the maturity of the option.
For instance, a \textit{fixed-strike} Asian call has $g(x, a) := e^{-rT} \left[ a - K \right]_+$ where $K$ denotes the strike
and a \textit{floatting-strike} Asian call has $g(s, a) := e^{-rT}  \left[ s - a \right]_+$.
In the first case, the option is exercised by its owner
if the underlying has lied above the strike \textbf{on average} throughout its lifetime. In the second case,
it is worth exercising it when the underlying is above its average value at expiry.

\

This work studies different pricing techniques for Asian options and will tackle \textbf{only fixed-strike Asian calls}
for simplicity. Furthermore, we will use Black-Scholes model. In that context, simulating $S_T$ is straightforward
and the real challenge consists in simulating $A_T$. That is why our developments for fixed-strike Asian calls
adapt directly to any Asian option of the form given above.

\

We first introduce and implement different Monte-Carlo approaches as developed by Lambert et al. \cite{main}
and B. Bouchard \cite{Bouchard}. We then compare them with a PDE approach \textit{(to be chosen)}.

\section{Naive approach}

The most basic Monte-Carlo approach to the problem consists in approximating the integral
of the underlying over its trajectory by a Riemann sum.

\[
	A_T \approx \bar A_T^{r, m} := \frac{h}{T} \sum_{k=0}^{m-1} \bar S_{t_k}^m
	= \frac{1}{m} \sum_{k=0}^{m-1} \bar S_{t_k}^m \quad \text{with} \quad h = \frac{T}{m}
\]

where $\bar S_{t_k}^m$ denotes the $k$-th step of an Euler scheme. That gives the following estimate:

\begin{equation}
	C_r := e^{-rT} \sum_{i=1}^n \left[ \frac{1}{m} \sum_{k=0}^{m-1} \bar S_{t_k}^{m, i} - K \right]_+
	\tag{1}
\end{equation}

where $\left( \bar S^{m, i} \right)_{i = 1, \dots, n}$ are independent and identically distributed copies of $\bar S^m$.
%According to B. Bouchard \cite{Bouchard}, scheme (1) has a weak error in $O \left( \frac{1}{n} \right)$.

\section{Improved Monte-Carlo approaches}
\subsection{Two finer approximations}

A first improvement of the above approach proposes to better use the information provided by the simulation
$\bar S_0^m, \bar S_{t_1}^m, \dots, \bar S_T^m$ to approximate the integral $A_T$.
It relies on the fact that once the trajectory has been simulated, the best estimation of the price is:
\[
	\bar C^m := \mathbb E \left[ e^{-rT} \left[ A_T - K \right]_+ \mid \bar S_0^m, \bar S_{t_1}^m, \dots, \bar S_T^m \right]
\]

By the tower property, the expectation (estimated by a Monte-Carlo method with $n$ trajectories)
of this conditional expectation is the price of the Asian call.
At this stage, this quantity is not known either. \cite{main} introduces the following simplification:

\begin{equation}
	\bar C^m \approx e^{-rT}  \left[ \mathbb E \left[A_T
		\mid \bar S_0^m, \bar S_{t_1}^m, \dots, \bar S_T^m \right] - K \right]_+
	\tag{S}
\end{equation}

A first-order Taylor expansion leads to the additional approximation below:
\begin{align*}
	\mathbb E \left[ \int_{t_k}^{t_{k+1}} S_u du \mid \bar S_0^m, \bar S_{t_1}^m, \dots, \bar S_T^m \right]
	&= \int_{t_k}^{t_{k+1}} \mathbb E \left[ S_u \mid \bar S_{t_k}^m, \bar S_{t_{k+1}}^m \right] du \\
	&= \bar S_{t_k}^m \int_{t_k}^{t_{k+1}} \mathbb E \left[ \frac{S_u}{\bar S_{t_k}^m}
		\ \Big\vert \ \bar S_{t_k}^m, \bar S_{t_{k+1}}^m \right] du \\
	&\approx
		\bar S_{t_k}^m \int_{t_k}^{t_{k+1}} \left\{ 1 + \ln \mathbb E \left[ \frac{S_u}{\bar S_{t_k}^m}
		\ \Big\vert \ \bar S_{t_k}^m, \bar S_{t_{k+1}}^m \right] \right\} du
\end{align*}
for all $k \in \{ 0, \dots, m - 1 \}$. On the one hand, Black-Scholes model gives:
\[
	S_u \mid \bar S_{t_k}^m, \bar S_{t_{k+1}}^m = \bar S_{t_k}^m \exp
	\left\{ \left( r - \frac{\sigma^2}{2} \right) (u - t_k) + \sigma
	\left ( \left( W_u \mid \bar W_{t_k}^m, \bar W_{t_{k+1}}^m \right) - \bar W_{t_k}^m \right) \right\}
\]
where $\bar W_{t_k}^m$ is the $k$-th step of the Brownian Motion used to simulate $\bar S^m$ as
part of the Euler scheme, ie:
$\bar S_{t_k}^m = S_0 \exp \{ ( r - \frac{\sigma^2}{2} ) t_k + \sigma \bar W_{t_k}^m \}$.
On the other hand, $W_u \mid \bar W_{t_k}^m, \bar W_{t_{k+1}}^m$ follows a Brownian Bridge.
Therefore, the integrand has an analytical expression indeed and it yields the following scheme:

\begin{equation}
    A_T \approx \bar A_T^{e, m} = \frac{1}{m} \sum_{k=0}^{m-1} \bar S_{t_k}^m
    	\left( 1 + \frac{rh}{2} + \sigma \frac{\bar W_{t_{k+1}}^m - \bar W_{t_k}^m}{2} \right) \tag{2}
\end{equation}

The above development is actually equivalent to a trapezoidal method in comparison with the more
basic Riemann sum used in scheme $(1)$. We note $C_e$ the corresponding estimate of the price.

\

Instead of simplification (S), \cite{Bouchard} and \cite{main} suggest a quite similar approach. For each step
of the Monte-Carlo estimation, first fix a trajectory with the explicit formula given by Black-Scholes model
(same as before). Then, rather than computing the conditional expectation $\bar C^m$, simulate a realization of
$e^{-rT} \left[ A_T - K \right]_+$ conditionally to the trajectory. Similarly:
\begin{align*}
	\int_{t_k}^{t_{k+1}} S_u du
	&= S_{t_k} \int_{t_k}^{t_{k+1}} \exp \left\{ \left( r - \frac{\sigma^2}{2} \right) (u - t_k) + \sigma
	\left( W_u - W_{t_k} \right) \right\} du \\
	&\approx
	S_{t_k} \int_{t_k}^{t_{k+1}} \left\{ 1 + r (u - t_k) + \sigma \left( W_u - W_{t_k} \right) \right\} du \\
	&= h S_{t_k} \left\{ 1 + \frac{rh}{2} + \frac{\sigma}{h} \int_{t_k}^{t_{k+1}} \left( W_u - W_{t_k} \right) du \right\}
\end{align*}

Furthermore, the remaining integral of the increment of the Brownian Motion is a Gaussian variable and we can
compute its expectation and variance conditionally to the trajectory since the integrand follows a Brownian Bridge.
Thus, we can indeed simulate it as stated above. In the following, we note:
\[
	\bar I_k^m := \frac{1}{h}
	\int_{t_k}^{t_{k+1}} \left( W_u - W_{t_k} \right) du \ \Big\vert \ \bar W_0^m, \bar W_{t_1}^m, \dots, \bar W_T^m
\]

The above finally yields the following scheme:

\begin{equation}
    A_T \approx \bar A_T^{p, m} = \frac{1}{m} \sum_{k=0}^{m-1}
    	\bar S_{t_k}^m \left( 1 + \frac{rh}{2} + \sigma \bar I_k^m \right) \tag{3}
\end{equation}

\begin{algorithm}[hbt!]
\caption{Scheme (3) implementation}
\KwData{$n$ (number of independent simulations), $m$ (number of time steps)}
\KwResult{Estimation and $95\%$ confidence interval}
\For{$i = 1, \dots, n$}{
	Simulate $\bar W^{m, i}$\;
	Deduce $\bar S^{m, i}$ using Black-Scholes formula\;
	\For{$k=0, \dots, m - 1$}{
		Compute the mean and variance of $\bar I_k^m$ conditionally to $\bar W^{m, i}$\;
		Simulate $\bar I_k^m$ accordingly\;
		$\dots$\;
	}
}
\Return{$\dots$}\;
\end{algorithm}

\subsection{The use of a control variable}

In order to improve the convergence speed, Lambert et al. \cite{main} finally propose a variance reduction technique
for the three schemes above. It uses a control variable as introduced by Kemna et al. \cite{Vorst}.

\begin{equation}
	\theta = \sum_{i=1}^n \left( e^{-rT} \left[ A_T^i - K \right]_+ + \beta \left( Z^i - \mathbb E [Z] \right) \right)
	\tag{$\ast$}
\end{equation}

Observing that $e^x \approx 1 + x$ and $\ln(1 + x) \approx x$ when $| x |$ is small,
the idea relies on the approximation:
\[
	A_T = \frac{1}{T} \int_0^T S_u du \approx \exp \left\{ \frac{1}{T} \int_0^T \ln S_u du \right\}
	= S_0 \exp \left\{ \frac{1}{T} \int_0^T \ln \frac{S_u}{S_0} du \right\}
\]
The equality on the right-hand side justifies the validity of such approximation: if $r$ and $\sigma$ are small,
$S_u$ can be expected to remain near $S_0$ and $\ln \frac{S_u}{S_0} \ll 1$.
Therefore, we would like to use the following as a control variable in the case of a fixed-strike Asian call:
\[
	Z
	= e^{-rT} \left[ S_0 \exp \left\{ \left( r - \frac{\sigma^2}{2} \right) \frac{T}{2} +
		\frac{\sigma}{T} \int_0^T W_u du \right\}
		- K \right]_+
\]

Since $\frac{1}{T} \int_0^T W_u du \sim \mathcal N \left(0, \frac{T}{3} \right)$, the exact expression
of $\mathbb E[Z]$ is indeed known.

However, the integral of $(W_t)_t$ needs itself to be estimated. That is why we define:
\begin{equation}
	\bar Z^{r, m} = e^{-rT} \left[ S_0 \exp \left\{ \left( r - \frac{\sigma^2}{2} \right) \frac{T}{2} +
		\frac{\sigma}{n} \sum_{i=0}^{n-1} \bar W_{t_k}^m \right\} - K \right]_+
	\tag{$i$}
\end{equation}

Plugging $\bar A_T^{r, m}$ from scheme $(1)$ and $\bar Z_T^{r, m}$ in $(\ast)$ yields a new scheme:
\begin{equation}
	\left[ \bar A_T^{r, m} - K \right]_+ + \hat\beta_r \left( \bar Z_T^{r, m} - \mathbb E [Z] \right)
	\tag{4}
\end{equation}
where $\hat\beta_r$ is estimated with the empirical correlation method.

\subsection{The combination of both improvements}

As we did with scheme $(1)$, we can plug the estimates of schemes $(2)$ and $(3)$ in $(\ast)$. That requires
to adapt the estimates of $Z$ too.
\begin{equation}
	\bar Z^{e, m} = e^{-rT} \left[ S_0 \exp \left\{ \left( r - \frac{\sigma^2}{2} \right) \frac{T}{2} +
		\frac{\sigma}{n} \sum_{i=0}^{n-1} \frac{\bar W_{t_{k+1}}^m + \bar W_{t_k}^m}{2} \right\} - K \right]_+
	\tag{$ii$}
\end{equation}
\begin{equation}
	\bar Z^{p, m} = e^{-rT} \left[ S_0 \exp \left\{ \left( r - \frac{\sigma^2}{2} \right) \frac{T}{2} +
		\frac{\sigma}{n} \sum_{i=0}^{n-1}
		\int_{t_k}^{t_{k+1}} \left( \bar W_u^m - \bar W_{t_k}^m \right) du
		\right\} - K \right]_+
	\tag{$iii$}
\end{equation}

It yields two new schemes.

\section{PDE approaches}

\section*{Conclusion}


\bibliography{bibliography}{}
\bibliographystyle{plain}

\end{document}